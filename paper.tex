%%%% IACR Transactions TEMPLATE %%%%
% This file shows how to use the iacrtrans class to write a paper.
% Written by Gaetan Leurent gaetan.leurent@inria.fr (2020)
% Public Domain (CC0)


%%%% 1. DOCUMENTCLASS %%%%
\documentclass[journal=tosc,submission]{iacrtrans}
%%%% NOTES:
% - Change "journal=tosc" to "journal=tches" if needed
% - Change "submission" to "final" for final version
% - Add "spthm" for LNCS-like theorems


%%%% 3. AUTHOR, INSTITUTE %%%%
\author{Heim, Maximilian}
\institute{
  Institute University Albstadt-Sigmaringen, Albstadt-Ebingen, Germany, \email{MaximilianHeim@protonmail.com}
}
%%%% NOTES:
% - We need a city name for indexation purpose, even if it is redundant
%   (eg: University of Atlantis, Atlantis, Atlantis)
% - \inst{} can be omitted if there is a single institute,
%   or exactly one institute per author


%%%% 4. TITLE %%%%
\title{An Introduction to Return Oriented Programming}
%%%% NOTES:
% - If the title is too long, or includes special macro, please
%   provide a "running title" as optional argument: \title[Short]{Long}
% - You can provide an optional subtitle with \subtitle.

\begin{document}

\maketitle


%%%% 5. KEYWORDS %%%%
\keywords{ROP, Return Oriented Programming, Exploit}


%%%% 6. ABSTRACT %%%%
\begin{abstract}
  In this paper we explore the possibilities of Return Oriented Programming
\end{abstract}


%%%% 7. PAPER CONTENT %%%%
\section{Introduction}

\section{Methods}

\section{Proof of concept}

\section{Conclusion}

\section{Main Result}
\label{sec:main}


%%%% 8. BILBIOGRAPHY %%%%
\bibliographystyle{alpha}
\bibliography{abbrev3,crypto,biblio}
%%%% NOTES
% - Download abbrev3.bib and crypto.bib from https://cryptobib.di.ens.fr/
% - Use bilbio.bib for additional references not in the cryptobib database.
%   If possible, take them from DBLP.

\end{document}
